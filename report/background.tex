\section{Background and Prior Works}

A large body of prior work has explored that eye contact is a critical aspect of human communication. [1, 2] Eye contact plays an important role in both in person and a WVC system. [3, 4] Therefore, it’s critical and necessary to preserve eye contact in order to realistically imitate real-world communication in WVC systems. However, perceiving eye contact is difficult in existing video-conferencing systems and hence limits their effectiveness. [2] The lay-out of the camera and monitor severely restricted the support of mutual gaze. Using current WVC systems, users tend to look at the face of the person talking which is rendered in a window within the display(monitor). But the camera is typically located at the top of the screen. Thus, it’s impossible to make eye contact. People who use consumer WVC systems, such as Zoom, Skype, experience this problem frequently. This problem has been around since the dawn of video conferencing in 1969 [5] and has not yet been convincingly addressed for consumer-level systems.

Some researchers aim to solve this by using custom-made hardware setups that change the position of the camera using a system of mirrors [6,7]. These setups are usually too expensive for a consumer-level system. Software algorithms solutions have also been explored by synthesizing an image from a novel viewpoint different from that of the real camera. This method normally proceeds in two stages, first they reconstruct the geometry of the scene and in second stage, they render the geometry from the novel viewpoint. [8, 9, 10, 11, 12] Those methods usually require a number of cameras and not very practical and affordable for consumer-level. Besides, those methods also have a convoluted setup and are difficult to achieve in real-time.

Some gaze correction systems are also proposed, targeting at a peer- to-peer video conferencing model that runs in real-time on average consumer hardware and requires only one hybrid depth/color sensor such as the Kinect. [13] However, when there are more than two persons involved in a web video conference, even with gaze corrected view, users still cannot tell whether a person is looking at him or someone else in the meeting. With the gaze correction, it will create the illusion that everyone in this meeting is looking out of the screen. This could cause a serious confusion.


\subsection{Eye Contact in Multi-person Conversation}

Most studies of eye contact during conversations focused on two-person communication [14]. 
However, multi-person conversational structure becomes more complicated when a third speaker is introduced. It has long been presumed that eye contact provides critical information in conversations. Isaacs and Tang [15] performed a usability study of a group of five participants using a desktop video conferencing system. They found that during video conferencing, users addressed each other by name and started explicitly requesting individuals to start talking. In face-to-face interaction, they found people used their eye gaze to indicate whom they were addressing. Sellen [16] was one of the first to formally investigate the effects of eye contact on the turn taking process in four-person video conferencing. Unfortunately, she found no effects because the video conferencing system she implemented did not accurately convey eye contact [16]. Vertegaal et al. [17] found that without eye contact, 88\% of the participants indicated they had trouble perceiving whom their partners were talking to.

Therefore, we propose our system

NOTE: ALL References can be found in Latex


`insert existing solution`

`insert VR 3d chat apps`

`insert other research work`
